% Cover letter using letter.sty
\documentclass{letter} % Uses 10pt
%Use \documentstyle[newcent]{letter} for New Century Schoolbook postscript font
% the following commands control the margins:
\topmargin=-0.5in    % Make letterhead start about 1 inch from top of page 
\textheight=8in  % text height can be bigger for a longer letter
\oddsidemargin=0pt % leftmargin is 1 inch
\textwidth=6.5in   % textwidth of 6.5in leaves 1 inch for right margin

\usepackage{graphicx}
\usepackage{wallpaper}
\ULCornerWallPaper{1}{figures/CSM_Letterhead.pdf}

\begin{document}

%\signature{Steven C. DeCaluwe}           % name for signature 
\longindentation=0pt                       % needed to get closing flush left
\let\raggedleft\raggedright                % needed to get date flush left
 
 
\begin{letter}{ }

\vspace{-20mm}
\begin{flushleft}
{\large\bf Steven C. DeCaluwe}
\end{flushleft}
\medskip\hrule height 1pt
\begin{flushright}
\hfill Associate Professor\\
\hfill Department of Mechanical Engineering \\
\hfill Colorado School of Mines \\
%\hfill 1601 Illinois St. \\
%\hfill Golden, CO  80401\\
\hfill decaluwe@mines.edu\\
\end{flushright} 
%\vfill % forces letterhead to top of page

\opening{Dear Editors at the {\it Journal of the Electrochemical Society},} 

\noindent Please find attached our manuscript ``Predicted Impacts of Pt and Ionomer Distributions on Low Pt-loading PEMFC Performance,'' submitted for publication in \emph{The Journal of the Electrochemical Society}. 

The paper presents a pseudo-2D model, used to understand and mitigate limiting phenomena in low-cost proton exchange membrane fuel cells (PEMFCs). Low-cost, high-performance PEMFCs are currently limited by poor understanding and control of interrelated physiochemical processes (species transport and electrochemical reactions) in the cathode catalyst layer (CL).  

Here, we use novel structure-property relationships for transport in Nafion thin films in the CL, which are responsible for moving reactants and products to and from Pt catalyst surfaces. The resulting transport parameters allow accurate fits to experimental PEMFC data with varying Pt loading, enabling new insights into limiting phenomena for low-Pt-loaded PEMFCs. Results demonstrate that, although the performance losses are driven by kinetic limitations, they manifest as increasing Ohmic losses. %Comparing to previous PEMFC model results, we demonstrate the impact of accurate transport property estimates to identify PEMFC limits.

Based on this improved understanding of low-cost PEMFC losses, we present a systematic study of CL design parameters (CL thickness and Nafion and Pt loading). Results highlight design principles to improve power and limiting currents in low-Pt PEMFCs, and identify a novel CL design — where Pt and Nafion loading both vary as a function of CL depth — that is predicted to increase maximum power density by nearly 25\%.

The paper therefore contributes new understanding and establishes important PEMFC design principles and modeling priorities to guide development of high performance energy conversion devices of the future. Moreover, the modeling tools are made available as open source git repositories for adoption by interested parties. We feel the manuscript would be of great interest to the community of \emph{JES} readers.

\closing{Sincerely,\\[-0.5mm]
\fromsig{\includegraphics[scale =0.5, trim = 16mm 6.75in 0 3.4in, clip]{figures/signature.pdf}}\\[1mm]

\fromname{Steven C. DeCaluwe\\
\vspace{3.0mm}
On behalf of:\\
Corey R. Randall}
} 





%Steven C. DeCaluwe

%On behalf of:
%Daniel Korff
%Andrew M. Colclasure
%Yeyoung Ha
%Kandler A. Smith



%\noindent 
% \closing{Sincerely yours, 
% \vspace{3mm}
%Enclosures

\end{letter}
 

\end{document}






